%%%%%%%%%%%%%%%%%%%%%%%%%%%%%%%%%%%%%%%
% Deedy - One Page Two Column Resume
% LaTeX Template
% Version 1.2 (16/9/2014)
%
% Original author:
% Debarghya Das (http://debarghyadas.com)
%
% Original repository:
% https://github.com/deedydas/Deedy-Resume
%
% IMPORTANT: THIS TEMPLATE NEEDS TO BE COMPILED WITH XeLaTeX
%
% This template uses several fonts not included with Windows/Linux by
% default. If you get compilation errors saying a font is missing, find the line
% on which the font is used and either change it to a font included with your
% operating system or comment the line out to use the default font.
%
%%%%%%%%%%%%%%%%%%%%%%%%%%%%%%%%%%%%%%
%
% TODO:
% 1. Integrate biber/bibtex for article citation under publications.
% 2. Figure out a smoother way for the document to flow onto the next page.
% 3. Add styling information for a "Projects/Hacks" section.
% 4. Add location/address information
% 5. Merge OpenFont and MacFonts as a single sty with options.
%
%%%%%%%%%%%%%%%%%%%%%%%%%%%%%%%%%%%%%%
%
% CHANGELOG:
% v1.1:
% 1. Fixed several compilation bugs with \renewcommand
% 2. Got Open-source fonts (Windows/Linux support)
% 3. Added Last Updated
% 4. Move Title styling into .sty
% 5. Commented .sty file.
%
%%%%%%%%%%%%%%%%%%%%%%%%%%%%%%%%%%%%%%%
%
% Known Issues:
% 1. Overflows onto second page if any column's contents are more than the
% vertical limit
% 2. Hacky space on the first bullet point on the second column.
%
%%%%%%%%%%%%%%%%%%%%%%%%%%%%%%%%%%%%%%


\documentclass[]{deedy-resume}
\usepackage{fancyhdr}
%\usepackage[style=authoryear]{biblatex}

\pagestyle{fancy}
\fancyhf{}

\rfoot{Page \thepage \hspace{1pt}}
\begin{document}

%%%%%%%%%%%%%%%%%%%%%%%%%%%%%%%%%%%%%%
%
%     LAST UPDATED DATE
%
%%%%%%%%%%%%%%%%%%%%%%%%%%%%%%%%%%%%%%
\lastupdated

%%%%%%%%%%%%%%%%%%%%%%%%%%%%%%%%%%%%%%
%
%     TITLE NAME
%
%%%%%%%%%%%%%%%%%%%%%%%%%%%%%%%%%%%%%%
\namesection{Lucas}{Rod\'es-Guirao}{ 
\urlstyle{same}\url{https://lcsrg.me} | \href{mailto:hi@lcsrg.me}{hi@lcsrg.me}% | \href{tel:+46 765 772 056}{+34 (0) 765~772~056}
}
%%%%%%%%%%%%%%%%%%%%%%%%%%%%%%%%%%%%%%
%
%     COLUMN ONE
%
%%%%%%%%%%%%%%%%%%%%%%%%%%%%%%%%%%%%%%

\begin{minipage}[t]{0.54\textwidth}
\sectionsep
%\vspace*{5pt}

%%%%%%%%%%%%%%%%%%%%%%%%%%%%%%%%%%%%%%
%     EXPERIENCE
%%%%%%%%%%%%%%%%%%%%%%%%%%%%%%%%%%%%%%
\section{Experience}

\runsubsection{eDreams Odigeo} \\
\descript{Data Scientist}
\location{June 2018 - present | Barcelona, Spain}
\textsec{Applying Machine Learning in Product, Content and more areas.}
\sectionsep

\runsubsection{National Institute of Informatics} \\
\descript{Research Intern}
\location{November 2017 - May 2018 | Tokyo, Japan}
\textsec{Deep Learning for \href{http://agora.ex.nii.ac.jp/~kitamoto/research/typhoon/}{Digital Typhoon Project} under supervision of  \href{http://www.nii.ac.jp/en/faculty/digital_content/kitamoto_asanobu/}{Prof. Kitamoto} (Master Thesis).}
\sectionsep

\runsubsection{Tracy} \\
\descript{Software Developer}
\location{April 2017- March 2018 | Stockholm, Sweden}
\textsec{AI/Machine learning based algorithm for pet-health IT device.}
\sectionsep

\runsubsection{KTH Royal Institute of Technology} \\
\descript{Teaching Assistant}
\location{January 2017 - October 2017 | Stockholm, Sweden}
\textsec{Assisting \href{http://www.csc.kth.se/~atsuto/}{\textbf{Prof. Atsuto Maki}} in \href{https://www.kth.se/social/course/DD2431/}{Machine Learning Course}.}
%    \item Member of a Research lab with 15 members.
\sectionsep

%\runsubsection{Cisco Systems}
%\descript{Incubator Program FY16}
%\location{Nov 2015 -�� Feb 2016 | Barcelona, Spain}
%EMEAR Engineer Incubator Program FY16
%\sectionsep

%\runsubsection{Fira Barcelona}
%\descript{Volunteer Guide }
%\location{November 2015 | Barcelona, Spain}
%Smart City Expo World Comgres (SCEWC) 2015
%\sectionsep

\runsubsection{Karlsruhe Institute of Technology} \\
\descript{Research Assistant}
\location{March 2015 - July 2015 | Karlsruhe, Germany}
\textsec{Cognitive Radio under supervision of \textbf{\href{https://www.cel.kit.edu/english/team_jondral.php}{Univ.-Prof. Dr. rer.nat. Friedrich K. Jondral}} and \textbf{\href{https://www.cel.kit.edu/english/team_1316.php}{Ankit Kaushik}} (Baschelor Thesis).}

\sectionsep


%%%%%%%%%%%%%%%%%%%%%%%%%%%%%%%%%%%%%%
%     EDUCATION
%%%%%%%%%%%%%%%%%%%%%%%%%%%%%%%%%%%%%%

\section{Education}

\runsubsection{KTH Royal Institute of Technology} \\
\descript{M.Sc in Electrical Engineering}
\location{Sept 2015 - June 2018  | Stockholm, Sweden}
\textsec{ Cum GPA: 3.64/4 }
%\location{ Cum. GPA: 8.76 / 10.00 }
\sectionsep


\runsubsection{Polytechnic University of Catalonia} \\
\descript{M.Eng in Telecommunications} 
\location{ Sept 2015 - June 2018  | Barcelona, Spain}
\textsec{Cum GPA: 3.53/4}

\descript{B.Eng in Telecommunications}
\location{ Sept 2011 - Sept 2015 | Barcelona, Spain}
\textsec{Cum GPA: 3.3/4}
\sectionsep

\runsubsection{Deutsche Schule Barcelona} \\
\location{Sept 2003 - June 2011  | Barcelona, Spain}
\textsec{Abitur: 2.5 (High School)}
\sectionsep

%%%%%%%%%%%%%%%%%%%%%%%%%%%%%%%%%%%%%%
%     PUBLICATIONS
%%%%%%%%%%%%%%%%%%%%%%%%%%%%%%%%%%%%%%


\definecolor{dark5}{HTML}{666666}


\section{Publications}
\sectionsep
{\fontspec[Path = fonts/lato/]{Lato-Lig}\color{dark5} \small \fontsize{10pt}{12pt}
\renewcommand\refname{\vskip -1cm} % Couldn't get this working from the .cls file
\bibliographystyle{siam}
\bibliography{publications}}
\nocite{*}

\sectionsep

%\section{Research}
%\runsubsection{Square-Law Selector and Square-Law Combiner for Cognitive Radio Systems: An Experimental Study}
%\descript{\href{https://www.cel.kit.edu/download/Kaushik2_VTC16.pdf}{Online resource}}
%\location{\href{http://www.ieeevtc.org/vtc2016fall/}{Proceedings of IEEE 84th Vehicular Technology Conference (VTC), Sept 2016, Montreal, Canada}}
%Worked with \textbf{\href{https://www.cel.kit.edu/english/team_1316.php}{Ankit Kaushik}},  \textbf{\href{http://wwwen.uni.lu/snt/people/shree_krishna_sharma}{Dr.  Shree Krishna Sharma}}, \textbf{\href{http://wwwen.uni.lu/snt/people/symeon_chatzinotas}{Dr.  Symeon Chatzinotas}} and \textbf{\href{https://www.cel.kit.edu/english/team_jondral.php}{Prof.  Friedrich K. Jondral}}  to implement and analyse the performance of  multiple antenna devices in the context of Cognitive Radio.

\end{minipage}
%%%%%%%%%%%%%%%%%%%%%%%%%%%%%%%%%%%%%%
%
%     COLUMN TWO
%
%%%%%%%%%%%%%%%%%%%%%%%%%%%%%%%%%%%%%%
\hfill
\begin{minipage}[t]{0.45\textwidth}
\sectionsep
\sectionsep

%%%%%%%%%%%%%%%%%%%%%%%%%%%%%%%%%%%%%%
%     LINKS
%%%%%%%%%%%%%%%%%%%%%%%%%%%%%%%%%%%%%%

\section{Links}
{\fontspec[Path = fonts/lato/]{Lato-Lig}\fontsize{10pt}{12pt}
\begin{tabular}{lll}
\locationnb{GitHub} & \href{https://github.com/lucasrodes}{\bf @lucasrodes}\\
\locationnb{LinkedIn} & \href{https://www.linkedin.com/in/lucasrodes}{\bf @lucasrodes}\\
\locationnb{Twitter} & \href{https://www.twitter.com/lucasrodesg}{\bf @lucasrodesg}\\
\locationnb{Medium} & \href{https://medium.com/@lucasrg}{\bf @lucasrg}\\
\end{tabular}\\
}

%%%%%%%%%%%%%%%%%%%%%%%%%%%%%%%%%%%%%%
%     COURSEWORK
%%%%%%%%%%%%%%%%%%%%%%%%%%%%%%%%%%%%%%

\section{Coursework}

\runsubsection{Computer Science \& Electrical Engineering} \\
\textsec{Machine Learning, Deep Learning, Artificial Intelligence, Computer Vision, Algorithms, Information Retrieval, Unix Systems, Signals and Systems, Digital Signal Processing, Wireless Systems, Information Theory, Probability, Stochastic Processes.}
\sectionsep
 
\runsubsection{Online courses} \\
\textsec{Digital Signal Processing (Coursera), Python for Everybody (Coursera), Django for Everybody (Coursera), Deep Learning Specialization by DeepLearning.AI (Coursera), Convolutional Neural Networks for Visual Recognition (Stanford).}
\sectionsep
%Getting Started with Python \textbullet{} Python Data Structures \textbullet{} Using Python to Access Web Data \textbullet{}\\

%{\footnotesize \textit{\textbf{(Research Asst. \& Teaching Asst 2x) }}} \\

%%%%%%%%%%%%%%%%%%%%%%%%%%%%%%%%%%%%%%
%     SKILLS
%%%%%%%%%%%%%%%%%%%%%%%%%%%%%%%%%%%%%%

\section{Skills}
\textsec{Teamwork, project management, Good coding practices.}
\sectionsep

\runsubsection{Natural Languages} \\
{\fontspec[Path = fonts/lato/]{Lato-Lig}\fontsize{10pt}{12pt}
\begin{tabular}{lll}
\locationnb{Native:} & {\bf Spanish, Catalan}\\
\locationnb{Fluent:} & {\bf English, German}\\
\locationnb{Elementary:} & {\bf Swedish}\\
\end{tabular}\\
}
\sectionsep

\runsubsection{Programming Languages} \\
{\fontspec[Path = fonts/lato/]{Lato-Lig}\fontsize{10pt}{12pt}
\begin{tabular}{lll}
\locationnb{Advanced:} & {\bf Python, SQL}\\
\locationnb{Confident:} & {\bf Bash, SQL, JavaScript/TypeScript, Matlab}\\
\locationnb{Familiar:} & {\bf CSS, HTML, R, Rust, Java, C}\\
\end{tabular}\\
}
\sectionsep

\runsubsection{Tools \& Utilities} \\
\textsec{Git/Mercurial, ssh, docker, GitHub/Bitbucket, unit testing, A/B testing, GNU/Linux, CI/CD (Travis, Jenkins), Google Cloud Platform, numpy/scipy, pandas/dask, sklearn, tensorflow/keras, xgboost, plotly/bokeh, cookiecutter, GPU, React Native, Raspberry PI, JetBrains, VS Code, Jira/Confluence, Trello, Adobe Photoshop, Logic Pro, Blogging, DaVinci Resolve.}

\sectionsep

%%%%%%%%%%%%%%%%%%%%%%%%%%%%%%%%%%%%%%
%     EXTRA
%%%%%%%%%%%%%%%%%%%%%%%%%%%%%%%%%%%%%%

\section{Extra}

{\fontspec[Path = fonts/lato/]{Lato-Lig}\fontsize{10pt}{12pt}
\begin{tabular}{lll}
Member & \textbf{\href{www.telecos.cat}{Telecos.cat}} Regional Society\\
Guide & \textbf{\href{http://www.smartcityexpo.com/en/}{Smart City Expo World Congress 2015}}\\
Organiser & Youth Association enkbronats \\
& (Cabrera de Mar, Sept 2013 - Sept 2019) \\
Volunteering &  \textbf{\href{http://www.sjvietnam.org/}{SJ Vietnam}} (Hanoi, August 2014)\\
DJ/Producer & Freelance, Al-Faru and Be Water\\
Teacher & Maths/Physics for high school students\\
Open Source & pandas (contributor), whatstk (author)
\end{tabular}\\
}

\sectionsep
\textsecita{In my freetime I enjoy baking bread and playing the ukulele.}


\end{minipage}

%%%%%%%%%%%%%%%%%%%%%%%%%%%%%%%%%%%%%%%%%%%%%%%%%%%%%%%%%%%%%%%%%%%%%%%%%%%%%%%%%%%%%%
%% PROJECTS Page
%%%%%%%%%%%%%%%%%%%%%%%%%%%%%%%%%%%%%%%%%%%%%%%%%%%%%%%%%%%%%%%%%%%%%%%%%%%%%%%%%%%%%%

\namesection{Lucas}{Rod\'es-Guirao}{ 
\urlstyle{same}\url{http://lcsrg.me} | \href{mailto:hi@lcsrg.me}{hi@lcsrg.me}% | \href{tel:+46 765 772 056}{+34 (0) 765~772~056}
}


%%%%%%%%%%%%%%%%%%%%%%%%%%%%%%%%%%%%%%
%     COLUMN 1
%%%%%%%%%%%%%%%%%%%%%%%%%%%%%%%%%%%%%%
\begin{minipage}[t]{0.51\textwidth}
\sectionsep
\sectionsep
%\vspace*{5pt}

\section{Projects}
\sectionsep

\runsubsection{Odisi}
\descript{LAEN Technologies} 
\location{ Jan 2020 - May 2020  | Barcelona, Spain}
\textsec{\href{https://www.linkedin.com/company/laen/}{website}\\
Development of mobile App using React Native and TypeScript for iOS and Android.
}
\sectionsep

\runsubsection{whatsTK}
\descript{Open Source} 
\location{ Dec 2016 - Present  | Barcelona, Spain}
\textsec{\href{https://lcsrg.me/whatstk}{website} | \href{https://github.com/lucasrodes/whatstk}{github} | 
\href{https://pypi.org/project/whatstk}{pypi}\\
Development, testing, versioning, maintainance of an open source python library to read and analyse WhatsApp chat.
Distributed under the GPL-3.0 license.}
\sectionsep

\runsubsection{Hotel Booking Predictive Model}
\descript{eDreams ODIGEO} 
\location{ Feb 2019 - Jan 2020  | Barcelona, Spain}
\textsec{Implementation of Machine Learning model to predict the user propensity to book a hotel.}
\sectionsep

\runsubsection{Virtual Interlining}
\descript{eDreams ODIGEO} 
\location{ Sep 2018 - Jan 2019, Oct 2019 - Present  | Barcelona, Spain}
\textsec{\href{https://www.connectionreview.com/blog/virtual-interlining-in-aviation-what-it-means-exactly--30}{virtual interlining info}\\
Create new content combining flights/segments from different data sources (e.g. different carriers, providers etc.). I am in charge of the implementation and research of Machine Learning applied to \textit{Virtual
Interlining}, including exploratory data analysis, training/hyper-parameter tuning, productionalisation, testing,
versioning and maintainance of the model, analysis on the results from A/B tests, etc. Working with Data Engineer,
Backend Engineers and Business.}
\sectionsep

\runsubsection{Deep Learning for Digital Typhoon}
\descript{Academic (NII, KTH, UPC)} 
\location{ Nov 2017 - May 2018  | Tokyo, Japan}
\textsec{\href{https://www.diva-portal.org/smash/record.jsf?pid=diva2:1304600}{report} | 
\href{https://github.com/lucasrodes/pyphoon}{github} | 
\href{https://lcsrg.me/pyphoon}{documentation}\\
Application of Deep Learning and Computer Vision techniques to typhoon satellite image dataset. Design and implementation of (i) a classifier to differentiate tropical cyclones and extratropical cyclones and (ii) a regression model to estimate the centre pressure value of a typhoon. Part of my Master Thesis.
}
\sectionsep

\runsubsection{Deep Koalarization}
\descript{Academic (KTH Royal Insitute of Technology)} 
\location{ Jun 2017 - Dec 2017  | Stockholm, Sweden}
\textsec{\href{https://lcsrg.me/deep-koalarization}{website} | 
\href{https://github.com/baldassarreFe/deep-koalarization}{github} | \href{https://arxiv.org/abs/1712.03400}{paper}\\
\href{https://twitter.com/fchollet/status/917846097430638592?s=20}{Acknowledged} by Keras creator François Chollet, we
review recent approaches to colorize gray-scale images using deep learning methods and propose a new model: an
end-to-end Convolutional Neural Network (CNN) model with a CNN trained from scratch and high-level features extracted
from Inception-ResNet-v2 pre-trained model using Transfer Learning.
}
\end{minipage}
%%%%%%%%%%%%%%%%%%%%%%%%%%%%%%%%%%%%%%
%     COLUMN 2
%%%%%%%%%%%%%%%%%%%%%%%%%%%%%%%%%%%%%%
\hfill
\begin{minipage}[t]{0.48\textwidth}
\sectionsep
\sectionsep

\runsubsection{Sensor Motion Data Exploration}
\descript{Tracy Trackers} 
\location{ Apr 2017 - Mar 2018  | Stockholm, Sweden}
\textsec{\href{https://www.facebook.com/tracytrackers/}{website}\\
Development and implementation of an AI/Machine learning based algorithm for pet-health IT device in an early-stage 
Start-up. Data collection, data cleaning, Cloud architecture discussions, Start-up roadmap etc.}
\sectionsep

\runsubsection{Trump Twitter Analysis/Visualisation}
\descript{Academic (KTH Royal Insitute of Technology)} 
\location{ Apr 2017 - Jun 2017  | Stockholm, Sweden}
\textsec{\href{https://github.com/lovemarcus/Trump-Facts}{github} | 
\href{https://github.com/lovemarcus/Trump-Facts/blob/master/trump-facts-report.pdf}{report}\\
Visualization of President Trump Tweets using Kibana and ElasticSearch. This project was developed as part of the DD2476 Search Engines and Information Retrieval Systems course at KTH Royal Institute of Technology, spring 2017.}
\sectionsep

\runsubsection{Kernel PCA for Denoising}
\descript{Academic (KTH Royal Insitute of Technology)} 
\location{ Nov 2016 - Jan 2017  | Stockholm, Germany}
\textsec{\href{https://github.com/lucasrodes/kPCA-denoising-python/blob/master/docs/report.pdf}{report} |
\href{https://github.com/lucasrodes/kPCA-denoising-python}{github} |
\href{https://alex.smola.org/papers/1999/MikSchSmoMuletal99.pdf}{original paper}\\
We reproduced the experiments presented in the paper Kernel PCA and De-noising in Feature Spaces by Sebastian Mika, 
Bernhard Schölkopf, Alex Smola Klaus-Robert Müller, Matthias Scholz and Gunnar Rätsch as a project in DD2434 Machine
Learning Advance Course during Winter 2016.}
\sectionsep

\runsubsection{Book Author Classification}
\descript{Academic (Polytechnic University of Catalonia)} 
\location{ May 2016 - Jun 2016  | Barcelona, Spain}
\textsec{\href{https://github.com/lucasrodes/book-author-classifier/blob/master/docs/report.pdf}{report} |
\href{https://github.com/lucasrodes/book-author-classifier}{github}\\
Benchmarking of NLP classification models applied to spanish literature book fragments.}
\sectionsep

\runsubsection{Deployment of Energy Detector for Cognitive Relay with Multiple Antennas}
\descript{Academic (Karlsruhe Institute of Technology)} 
\location{ Mar 2015 - Jul 2015  | Karlsruhe, Germany}
\textsec{\href{https://upcommons.upc.edu/handle/2117/77499}{report}\\
Bachelor thesis in Telecommunications Engineering as an exchange student at KIT.\\ 
Deployment of a signal detector based on Energy Detection in order to take advantage of spectrum holes in a Cognitive 
Relay scenario. Furthermore, the case with multiple antennas in reception (i.e. where the detection process occurs) are
studied.
}

\end{minipage}

\end{document} \documentclass[]{article}
